\chapter{Service}
\label{l.service}

\begin{kcmt62}
\begin{komacmt62}
	Service bezieht sich nicht speziell auf einzelne Veranstaltungen, sondern
	aufs Studium allgemein. Wenn wegen Bachelor und Master die Mobilit"at
	zwischen Hochschulen und/oder L"andern steigt, ist die Beratung ein
	Minimalstandard!

\end{komacmt62}
\end{kcmt62}
\begin{kcmt}\begin{komacmt}
	\paragraph{Bearbeitete Punkte:}
	\begin{itemize}
		\item Studienberatung / Fachberatung
		\item Erstsemesterinformation
		\item Fachschafts-Service
		\item Auslandsangebot
	\end{itemize}

	\paragraph{Ausstehende Punkte:}
	\begin{itemize}
		\item Pr"a-Studierenden-Info
		\item Schulungen f"ur Studentische Hilfskr"afte (Tutoren etc.)
		\item "Offnungszeiten (Bibliothek, Sekretariat)
	\end{itemize}

	Service bezieht sich nicht speziell auf einzelne Veranstaltungen, sondern
	aufs Studium allgemein. Wenn wegen Bachelor und Master die Mobilit"at
	zwischen Hochschulen und/oder L"andern steigt, ist die Beratung ein
	Minimalstandard!
\end{komacmt}\end{kcmt}

\section{Studienberatung}

\begin{kcmt}\begin{komacmt}
\emph{Hochschulweite Studienberatung}
\begin{itemize}
	\item bei Fachfragen sofort \& richtig (Fachberatung) weiterleiten
	\item generellen "Uberblick bieten
\end{itemize}
\end{komacmt}\end{kcmt}

Es gibt eine hochschulweite Beratungszentrale, die kompetent ber"at und weiterleitet.
Das Beratungsangebot umfasst folgende Bereiche:
\begin{itemize}
	\item fachliche Beratung "uber die einzelnen Studieng"ange
	\item Studienfinanzierung
	\item Studienrechtsberatung
	\item Beratung f"ur
		\begin{itemize}
			\item behindertengerechtes Studium
			\item Studenten mit Kind
			\item ausl"andische Studenten
			\item Auslandsstudium
		\end{itemize}
\end{itemize}

Unter angemessener finanzieller und organisatorischer Unterst"utzung kann ein 
Teil der Beratungsverpflichtung an die organisierte Studierendenschaft abgetreten werden.

\begin{kcmt}\begin{komacmt}
	Das hei"st aber auch, dass falls Bereiche der Beratung an die
	Studierendenschaft abgetreten werden, muss die abtretende Stelle
	auch daf"ur Sorge tragen, dass die Beratung auch effizient
	durchgef"uhrt wird.
\end{komacmt}\end{kcmt}

\section{Fachberatung}

\begin{kcmt}\begin{komacmt}
\textbf{Fachberatung}
\begin{itemize}
	\item Studienplanung
	\item Pr"ufungsplanung
	\item "Uberblick "uber m"ogliche Studienvertiefung(en)
	\item Anerkennung von Leistungen von anderen Hochschulen
	\item Informationen zum Studienwechsel
\end{itemize}
\end{komacmt}\end{kcmt}

Die Fachberatung ist daf"ur zust"andig, dass ein Student sein Studium
zielgerichtet durchf"uhren kann. Sie muss insbesondere zu folgenden
Themen kompetent beraten k"onnen:
\begin{itemize}
	\item Studienplanung
	\item Pr"ufungsplanung
	\item Studienvertiefung(en)/Spezialisierung(en)
	\item g"angige Nebenf"acher
	\item Anerkennung von Leistungen, die an anderen Hochschulen erbracht wurden
	\item Studienwechsel, sowohl Wechsel des Studiengangs als auch der Hochschule
\end{itemize}

Innerhalb der Vorlesungszeit ist eine Beratung sp"atestens eine Woche nach
Anfrage eines Studenten gew"ahrleistet. In der vorlesungsfreien Zeit
kann diese Frist auf allerh"ochstens drei Wochen verl"angert werden.

\section{Betreuung der Studienanf"anger}

Jeder Studienanf"anger wird vor Studienbeginn "uber das Informationsveranstaltungsangebot orientiert.
Dieses beinhaltet ein Infoheft sowie die M"oglichkeit einer pers"onlichen Beratung.

\begin{description}
	\item [Beratung] Die persönliche Beratung bietet vor allem einen Ausblick "uber das Studium, informiert "uber Voraussetzungen und Fristen und leitet bei weiterf"uhrenden Fragen an die entsprechenden Beratungsstellen weiter.

	\item [Infoheft] Das Infoheft (digital oder gedruckt) beinhaltet mindestens folgende Punkte:
	\begin{itemize}
		\item Pflichtveranstaltungen des ersten Jahres 
		\item Vorlesungskommentar zu Vorlesungen des ersten Semesters
		\item Wichtige Ansprechpartner bzw. Anlaufstellen (mit Telefonnummer,
			E-Mail, Raumnummer, Sprechzeiten wenn m"oglich)
		\item wichtige Termine (z.\,B. R"uckmeldungsfristen, Pr"ufungsanmeldungszeitr"aume)
		\item Infrastruktur (Lageplan, Rechnerzugang, "Offnungszeiten, Bibliothek)
	\end{itemize}

	\begin{kcmt}\begin{komacmt}
	Voraussetzungen \& Fristen: Was muss ich vor dem Studium noch leisten,
	wof"ur mich noch anmelden.
	\end{komacmt}\end{kcmt}
\end{description}

\section{Praktikum}

Falls in einem Studiengang ein Pflichtpraktikum vorgesehen ist, bietet die Hochschule eine Anlaufstelle f"ur Praktikumsbelange. Diese erm"oglicht eine zeitnahe Betreuung des Studenten. Insbesondere bietet sie regelm"a"sige Sprechzeiten an.

Die Anlaufstelle dient der Vermittlung von Praktikumsstellen. Sie bietet Informationen und Hilfestellung bei eventuellen Problemen. Dies umfasst Informationen rein organisatorischer Art wie auch praktikumsvorbereitende und praktikumsnachbereitende Informationen und Hilfestellungen.

Des Weiteren besitzt die Anlaufstelle Informationen zu eventuellen Fristen, Terminen oder Zulassungsvoraussetzungen und setzt Betroffene fr"uhzeitig dar"uber in Kenntnis.

\begin{kcmt}\begin{komacmt}
gel"oscht, weil in anderen Abschnitten bereits vorhanden

\section{Transparenz}
	\begin{itemize}
		\item "Uberblick "uber Vertiefungen in der Mathematik geben
		\item Was an meiner Hochschule, was wo anders
		\item Kooperation mit anderen Hochschulen
		\item Minimum "Uberblick auf Homepage mit Links
		\item besser Vortr"age, Ringvorlesung o.\,"a.
	\end{itemize}


Die Hochschule stellt jedem Studenten einen inhaltlichen "Uberblick "uber die m"oglichen
Vertiefungen in der Mathematik zur Verf"ugung. Hierbei sind die an dieser
Hochschule angebotenen Vertiefungen ausf"uhrlich dargestellt.

	Eine blo"se Aufz"ahlung reicht hier nicht.
	Wie kann das transportiert werden?
	\begin{itemize}
		\item Homepage
		\item Infoheft
		\item Vortr"age etc.
	\end{itemize}
\end{komacmt}\end{kcmt}

\section{Webseite}

Die Hochschule besitzt eine "offentlich zug"angliche, barrierefreie Webseite, "uber die mindestens folgende Informationen verf"ugbar sind:
\begin{itemize}
\item Aktuelle Pr"ufungsordnungen aller an der Hochschule existierenden Studieng"ange. Dazu geh"oren auch auslaufende Studieng"ange.

\item angebotene Lehrveranstaltungen inklusive Name des Dozenten, kurzer Beschreibung und etwaiger inhaltlicher Voraussetzungen

\item Liste der Lehrst"uhle und Professoren mit angegebener Kontaktm"oglichkeit

\item Verweis auf die Fachschaft und deren Webseite, sofern vorhanden

\item Verzeichnis der vorhandenen Servicestellen

\item wichtige Termine, wie zum Beispiel R"uckmeldefristen, Vorlesungszeiten, Pr"ufungsanmeldungszeitr"aume und -termine, Termine von Informationsveranstaltungen
\end{itemize}


\section{Auslandsangebot}

Die Hochschule bietet dem Studenten die M"oglichkeit eines Auslandsstudiums. Hierbei
unterst"utzt sie den Studenten bei der Wahl und dem Kontakt zu einer Austauschhochschule.

Ein Student, der von einer ausl"andischen Hochschule kommt, wird bez"uglich
Visa und anderen Rechtsfragen, Wohnungssuche, Integration und f"ur ihn geeignete Veranstaltungen
beraten und unterst"utzt.
