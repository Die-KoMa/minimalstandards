% $Id: praeambel.tex,v 1.9 2008/05/26 15:30:13 brainbug Exp $
\section{Pr"aambel}

In den letzten Jahren wurden im deutschsprachigen Raum rege
Debatten "uber die Qualit"at der Lehre gef"uhrt.
Bei diesen war der Einflu"s praxisferner Experten unserer Meinung nach "uberproportional gro"s.
Dadurch fehlten Standpunkte und Erfahrungen derer, die
die Praxis erleben, n"amlich unsere. Als Studierende
aller Semester, aller Hochschularten, aller mathematischen Studieng"ange
und aller auslaufenden oder eingef"uhrten Studiensysteme sehen
wir unsere Standpunkte im Ergebnis der Debatte nur unzureichend
ber"ucksichtigt.

Gleichzeitig muss man feststellen: Die Umsetzung des Reformprozesses
ist hinter den Erwartungen der Hochschulen, Wirtschaft und Politik
zur"uckgeblieben. Die gestiegene Belastung hindert die Studierenden
an der Entfaltung ihrer wissenschaftlichen F"ahigkeiten, worunter
die gesamte Wissenschaftslandschaft leidet und nachhaltig negativ
beinflusst wird.

Getrieben von einem Wunsch nach Eliten werden die Anforderungen der
Masse vernachl"assigt. Insbesondere wird ignoriert oder "ubersehen,
dass es das Ziel der Hochschulen sein muss, nicht nur eine Elite zu
schaffen, sondern das Gros erfolgreich zu Wissenschaftlern auszubilden.
Hierzu m"ussen gewisse Anforderungen von jeder Hochschule erf"ullt
sein, eben die im Folgenden aufgef"uhrten \emph{Minimalstandards}.
Die Bringschuld zur Erf"ullung dieser Minimalstandards liegt sowohl
bei den Hochschulen als auch bei Bund und L"andern, die die
"offentlichen Hochschulen zu finanzieren haben.

\paragraph{Anmerkungen zum Sprachgebrauch}\begin{enumerate}
\item Formulierungen im Pr"asens stellen Forderungen dar, die im
	Rahmen der Minimalstandards zu erf"ullen sind.
\item Zur besseren Lesbarkeit wurde auf die klare Geschlechterunterscheidung
	verzichtet. Die verwendeten Formulierungen richten sich
	jedoch ausdr"ucklich an beide Geschlechter.
\end{enumerate}
