\chapter{Infrastruktur}

\section{Generelles}

\begin{itemize}
\item Grundlegende Dinge, wie ausreichende
Beleuchtung, Heizung, Toiletten, Sitzm"oglichkeiten, Platz zum
Schreiben und auch Schreibmaterialien (z.B. Tafeln oder Whiteboards mit zugeh"origem Material) sind vorhanden.

\item Die Infrastruktur ist w"ahrend der Vorlesungszeiten f"ur den Studenten zug"anglich.

\item Alle R"aumlichkeiten sind barrierefrei zug"anglich.

\item Aufeinander folgende Pflichtveranstaltungen finden 
		nahe genug beieinander statt. Es ist also in der Zeit zwischen den 
		Veranstaltungen m"oglich,  von einem Veranstaltungsort zu dem der folgenden zu gelangen.

\item Geb"aude und R"aume sind deutlich 
		sichtbar (auch international verst"andlich) gekennzeichnet.
		An zentralen Stellen sind Pl"ane vorhanden.
\end{itemize}


\section{R"aume}

\subsection{Veranstaltungsr"aume}

\begin{kcmt}\begin{komacmt}
Vorlesungs- und Seminarr"aume unterscheiden sich nur in der Gr"o"se und 
werden deshalb nicht gesondert behandelt. Spezielle R"aume f"ur Tutorien u.~"a. 
werden hier nicht erw"ahnt, da diese nicht unbedingt erforderlich sind 
(Jede "Ubung kann auch in einem Vorlesungs-/Seminarraum stattfinden.). 
Wenn es extra "Ubungsr"aume g"abe, w"are die Anzahl der insgesamt 
ben"otigten R"aume gr"o"ser (kein Minimalstandard).
\end{komacmt}\end{kcmt}
\begin{itemize}
	\item F"ur jede Veranstaltung steht ein Raum zur Verf"ugung.
	\item Jeder Zuh"orer bekommt bei den Veranstaltungen einen daf"ur vorgesehenen Sitzplatz.
	\item Auch zu Sto"szeiten sind ausreichend Kapazit"aten an R"aumlichkeiten vorhanden.
	\item Die R"aume verf"ugen "uber eine Tafel, die so gro"s ist,
		dass bei einer f"ur alle Anwesenden lesbar gro"sen Anschrift gen"ugend Tafelfl"ache vorhanden ist, um den f"ur das Verst"andnis des aktuellen Themas notwendigen Kontext zu fassen.
	\item Die R"aumlichkeiten m"ussen die M"oglichkeit der Visualisierung per Beamer 
		und/oder Overheadprojektor bieten. Es gibt in jedem Raum eine Projektionsfl"ache. 
		Dazu ist jeder Raum (mindestens die H"alfte aller R"aume gleichzeitig) mit den ben"otigten Ger"aten versorgbar.
\begin{kcmt}\begin{komacmt}
	Es sind nicht zu wenig, weil fast nie alle R"aume gleichzeitig besetzt sind und auch
	f"ur viele Veranstaltungen kein Beamer / Overheadprojektor n"otig ist. Es sind nicht
	zu viel, da es nicht sein kann, da"s sich ein Vortragender in der Wahl der Visualisierung
	nach dem Vorhandensein von Beamer/Overheadprojektor richten mu"s.
\end{komacmt}\end{kcmt}
	\item In den R"aumen ist der Dozent "uberall zu verstehen,
		geeignete Hilfsmittel (z.B. ein Mikrofon) stehen bei Bedarf zur Verf"ugung.
	\item Es gibt Platz, um Jacken, Taschen usw. abzulegen.
\end{itemize}

\subsection{Computerr"aume}
\begin{itemize}
	\item Es gibt eine der Anzahl der Studenten angemessene Menge frei zug"anglicher Rechnerpl"atze.
\begin{kcmt}\begin{komacmt}
	Eine genauere Kl"arnung von "`zug"anglich"' wurde gew"unscht, aber auf eine Definition konnten wir uns noch nicht einigen. Es geht dabei z.B. um "Offnungszeiten, Wartezeiten, et al. Ein Vorschlag war, eine angemessene Anzahl an Rechnerpl"atzen dar"uber zu definieren, dass es Zeiten gibt, zu denen nicht gewartet werden muss.

	\paragraph{Gestrichene Formulierung:} \begin{itemize}
	\item Das Verh"altnis von Studierenden zu Rechnerpl"atzen gesamt unterschreitet 7:1 nicht.
	\end{itemize}

	"`Warum 7:1?"' -- Diese Zahl wurde aus der Situation an der Fachhochschule Regensburg,
	TU M"unchen und Chemnitz gefolgert. Die urspr"ungliche Forderung war hier mal 5:1.
	Das Plenum findet aber auch 10:1 akzeptabel. Hier stellt sich auch die Frage nach
	Me"sgr"o"sen allgemein: sollen sie minimal, maximal, exakt beschreiben? "`Wenn diese
	Zahl "uberschritten wird ist's \emph{definitiv} schlecht?"'

	Die neue Formulierung dr"uckt aus, da"s kleine Fakult"aten mit einer Anzahl von
	Rechner im Verh"altnis 10:1 deutlich zu wenig Rechner h"atten.
\end{komacmt}\end{kcmt}
	\item W"ahrend Veranstaltungen mit PC-Einsatz gibt es mindestens halb so viele Rechner wie Studenten.
	\item Die Rechner sind entsprechend der Richtung der Hochschule mit Software 
		ausgestattet (Computeralgebrasystem, Numerische Software, Statistikprogramm, \dots). 
		Ein Programm zum Anfertigen auch umfangreicherer mathematischer Texte ist installiert.
\begin{kcmt}\begin{komacmt}
	Diese Ausstattung ist nat"urlich einwandfrei lizensiert, legal erworben etc.
	Ob hier auch Lizenzen f"ur Spezialsoftware an Studenten herausgegeben werden
	wurde nicht explizit diskutiert (Rauschen aus dem Plena).
	Freeware bzw. Open Source Software sind nat"urlich auch erlaubt.
\end{komacmt}\end{kcmt}
	\item Die Rechner verf"ugen "uber einen Internetzugang. 
	\item Es gibt eine Druckm"oglichkeit h"ochstens zum Selbstkostenpreis. 
		Die Funktionsf"ahigkeit dieser ist immer gew"ahrleistet (Toner, Papier vorhanden).
	\item Es gibt eine M"oglichkeit zur Visualisierung bei Lehrveranstaltungen (z.~B. Beamer, Tafel...).
\end{itemize}

\subsection{Studentische Arbeitsr"aume}
\begin{itemize}
	\item F"ur den Studenten besteht die M"oglichkeit freie Kapazit"aten herauszufinden 
		(Raumbelegungsplan) und diese zu nutzen. 
	\item Es ist ein Ruhebereich vorhanden, in dem gearbeitet werden kann.
	\item Ein Student, der an seiner Abschlussarbeit arbeitet, hat immer Zugang zu einem Rechner. 
		(Es sollte immer mindestens ein Rechnerpool frei sein.)
	\item Studentische Arbeitsr"aume sind mit einer Tafel bzw. einem Whiteboard ausgestattet.
\end{itemize}


\section{Fachschaft}

Die Fachschaft ist ein Zusammenschluss von Studenten eines Faches zwecks gemeinsamer Interessenvertretung.
Die Fachschaft wird durch die Hochschule unterst"utzt.

\begin{kcmt}\begin{komacmt}
Es geht hier \emph{nicht} um die Aufgaben der Fachschaften.

\begin{itemize}
	\item Rechnerzugang inklusive Webspace \& Mail-Adresse(n)
	\item Kopierm"oglichkeit
	\item B"uroraum mit Telefon
	\item M"oglichkeit f"ur regelm"a"sige FS-Sitzungen
	\item Erm"oglichung der Herausgabe von Infomaterial
\end{itemize}
\end{komacmt}\end{kcmt}

Um eine effiziente Fachschaftsarbeit zu gew"ahrleisten, stellt die Hochschule der Fachschaft folgendes zur Verf"ugung:
\begin{itemize}
	\item einen B"uroraum mit Telefon, dessen Gr"o"se der Anzahl der Studenten angemessen ist,
	\item einen Rechnerzugang inklusive Webspace f"ur eine Fachschaftswebseite und
		eine E-Mail-Adresse sowie
	\item eine Kopierm"oglichkeit.
\end{itemize}

Au"serdem erm"oglicht die Hochschule der Fachschaft regelm"a"sige
Fachschaftssitzungen (durch Bereitstellen eines geeigneten Raumes) und die
Herausgabe von Infomaterial, z.\,B. den Druck eines regelm"a"sigen Infohefts, von Plakaten,
Flyern und "ahnlichem. 

\newpage
\section{Bibliothek}
Eine Bibliothek ist vorhanden und bietet mindestens folgende M"oglichkeiten:
\begin{itemize}
	\item ausreichende Recherchem"oglichkeiten (z.B. am Rechner)
	\item den Veranstaltungen zu Grunde liegende und vertiefende Literatur
	\item ein Kopierer
	\item Arbeitspl"atze
	\item Die wichtigsten Fachzeitschriften sind vor Ort vorhanden, die anderen 
		per Fernleihe beziehbar.
	\item Nicht vorhandene B"ucher sind per Fernleihe zu beziehen.
	\item Es findet regelm"a"sig eine "Uberpr"ufung des Bedarfs statt, 
		so dass bei h"aufig vergriffenen Werken der Bestand aufgestockt wird.
\end{itemize}
