\chapter{Veranstaltungsformen}
\thispagestyle{fancy}

\begin{kcmt}\begin{komacmt}
\textbf{M"ogliche Formen}...

\begin{itemize}
	\item Vorlesung: (definiert).
	\item "Ubung: (definiert).
	\item Tutorium: (definiert).
	\item Praktika: (Anforderungen bei Pflichtpraktika unter Infrastruktur zu finden).
	\item Seminar: (definiert)
	\item Fragestunden/Konversatorische Stunden (Braucht's nicht: T + "U + geile Studis = kein Problem).
	\item Nachhilfe (Braucht's nicht: T + "U + geile Studis = kein Problem).
	\item Vorkurs/Orientierung (kein Minimalstandard, n"aheres dazu siehe unten in den Kommentaren)
	\item Service (siehe "`Service"', S.~\pageref{l.service})
\end{itemize}

\end{komacmt}\end{kcmt}

\section{Globale Forderungen}

\begin{itemize}
\item	Alle Veranstaltungen der Hochschule sind frei zug"anglich. 
\begin{kcmt}\begin{komacmt}
	Mit "`frei zug"anglich"' ist gemeint, dass ein Student jede Veranstaltung zumindest als Gasth"orer besuchen darf. Eine aktive Teilnahme kann ggf. bei Veranstaltungsformen wie Seminaren, Praktika auf eine feste Teilnehmerzahl beschr"ankt sein. (neu: Graz KoMa 65)
\end{komacmt}\end{kcmt}

\item	Alle Lehrenden bieten w"ahrend ihrer Lehrveranstaltungen und der Betreuung von Pr"ufungsleistungen eine w"ochentliche Sprechstunde an oder zumindest die M"oglichkeit einen Termin innerhalb einer Woche zu vereinbaren. Sofern der Lehrende f"ur mehr als eine Woche nicht erreichbar ist, muss ein alternativer Ansprechpartner vorhanden sein.

\begin{kcmt}\begin{komacmt}
	(Karlsruhe) Im Teil-AK trat die Frage auf, ob eine garantierte w"ochentliche Sprechstunde
	nicht schon den Rahmen "`Minimal"' sprengt. Reicht nach Vereinbarung
	innerhalb einer Woche nicht schon aus?

	(Karlsruhe: Robert) ist eine w"ochentliche Sprechstunde nicht eine Einschr"ankung?
	Quasi gefordert, es reicht wenn der Prof nur zur Sprechstunde erreichbar ist?

	(Graz) Mit "`Termin innerhalb einer Woche vereinbar"' verstehen wir, dass die Terminabsprache nicht l"anger als eine Woche dauert, der Termin danach aber zeitnah stattfinden muss.
\end{komacmt}\end{kcmt}

\item	Die Lehrenden bzw. Betreuenden sind fachlich und didaktisch kompetent.
	
\item	Es werden nicht nur Probleml"osungen vermittelt. Es wird auch gelehrt, Probleme zu l"osen.

\begin{kcmt}\begin{komacmt}
	(Fritz) Hierbei ist gemeint, da"s nicht nur das reine Reproduzieren von bekannten
	L"osungen ("uberhaupt) eine Lehre der Mathematik auszeichnet, sondern das
	Vermitteln einer mathematischen Denke. Hierzu ist es absolut notwendig,
	da"s neben Probleml"osungen eben auch gelehrt wird Probleme zu l"osen.
\end{komacmt}\end{kcmt}
	
\item	Vom Studenten wird erwartet, den Stoff der vorhergehenden Lehrveranstaltung durch
	Aufbereitung ausreichend verinnerlicht zu haben, um ein kontinuierliches Voranschreiten
	im Stoff zu gew"ahrleisten. Der Zeitaufwand daf"ur "uberschreitet dabei das 
	eineinhalbfache der f"ur die Vorlesung vorgesehenen Zeit nicht.

\item	Die hier vorgestellten Veranstaltungsformen beziehen sich auf alle Phasen des Studiums.
	Der Gebrauch des Begriffes "`Basisveranstaltung"' beschreibt die Veranstaltungen der
	ersten Studienphase.
	
\item	Alle Veranstaltungen werden jedes Semester evaluiert, sofern die Anonymität der Befragten gewährleistet bleibt. Die Ergebnisse sind f"ur die Studenten in ausgewerteter Form zug"anglich.

\begin{kcmt}\begin{komacmt}
	Details zur Evaluation kommen anderswo her und ist auch von der Veranstaltungsform abh"angig.
	Nicht how-to-eval vorschreiben sondern da"s\dots

	(Graz) F"ur Details verweisen wir auf die Resolution der KoMa~64 zu Evaluationen.
\end{komacmt}\end{kcmt}

\item	Nach dem bestandenen ersten Studienabschnitt wird davon ausgegangen, dass alle Studenten 
	sich auf etwa gleichem Niveau befinden. Hierbei wird auch auf Schwankungen bei den
	Vorkenntnissen der Studenten eingegangen, d.h. das erreichte Niveau ist unabh"angig
	vom Zeitpunkt des Studienbeginns. Eventuell vorhandene und erkannte M"angel des Studenten
	werden durch zus"atzliche Veranstaltungen oder Hilfestellungen, wie z.\,B. "Ubungen, Zusatzmaterial ausgeglichen.

\item	F"ur die Mehrheit der Studenten gen"ugen 50 Stunden Arbeitsaufwand pro Woche, um das Studium in Regelstudienzeit erfolgreich abzuschlie"sen.
	Der Arbeitsaufwand beinhaltet sowohl die Zeit f"ur den Besuch von Veranstaltungen als auch f"ur die Nachbereitung, Hausaufgaben, schriftliche Arbeiten und "ahnliches.
\begin{kcmt}\begin{komacmt}
	x bestimmen. (Fritz) Hinweis: Eigentlich gibt es ECTS, nach derem System basierend auf
	Semesterwochenstunden "`ausgerechnet"' werden k"onnte, wieviel Wochenarbeitsstunden auf
	"Ubungsaufgaben entfallen k"onnen. Der Sinn hier ist eine Begrenzung nach oben, und es
	ist fraglich, ob die bearbeitende Gruppe einen Konsens mit dem ECTS findet.
\end{komacmt}\end{kcmt}
\end{itemize}

\section{Vorlesung}

\subsection{Definition} 
	Eine Vorlesung ist eine regelm"a"sige und fortlaufende Unterrichtsveranstaltung, die von einem
	Professor, Lehrbeauftragten oder wissenschaftlichen Mitarbeiter im Vortragsstil gehalten wird.


\subsection{Ziel} 
	Ziel von Vorlesungen ist die Vermittlung fachlichen Wissens auf theoretischer Basis. 

\subsection{Anforderungen} \label{vorlesung:anforderungen}

\begin{itemize}
	\item Der Lehrstoff ist inhaltlich und visuell so aufbereitet, dass die Studenten
	mehrheitlich nicht "uberfordert sind.
\begin{kcmt}\begin{komacmt}
	Hier betreten wir ein Minenfeld, das Spannungsfeld "`Qualit"at der Vorlesungen"' $\Leftrightarrow$ "`Qualit"at
	der Studierenden"'. Man k"onnte die Anforderung, Zwei Drittel der Studierenden nicht zu verlieren, auch
	dadurch erf"ullen, indem der Stoffumfang erheblich gek"urzt wird. Ref. Fachliche Breite und Tiefe :)

	Anders gesagt, der Stoff mu"s gleich bleiben bzw. der Stoffumfang sollte nicht gek"urzt werden
	um hier etwas zu erreichen.

	Weiterhin macht die Messung ein Problem: Gerade zu Beginn des Studiums sind einige Studierende
	noch anwesend, die f"ur das Studium (allgemein oder das der Mathematik) ungeeignet sind. In dieser
	inhomogenen Menge (bzgl. des vorherigen Ausbildungs- und Leistungsstand) eine Messung durchzuf"uhren
	f"uhrt hier am Ziel vorbei.

	Anmerkung aus dem Plenum: diese "`Messung"' kann ja auch von den betreuenden Studierenden
	durchgef"uhrt werden ($\rightarrow$ "Ubungsgruppenleiter, Mentor, \dots).
\end{komacmt}\end{kcmt}
	\item Durch Bereitstellung und/oder Verweise auf begleitende Lehrmaterialien ist es dem Studenten
		m"oglich das Lernziel auch autodidaktisch zu erreichen sowie in der Vorlesung angeeignetes Wissen weiter zu
		vertiefen.
	\item Eine Vorlesung wird bei Basisveranstaltungen grunds"atzlich von "Ubungen und/oder Tutorien begleitet. Das Verh"altnis der Stundenzahl von "Ubungen/Tutorien zur Vorlesung betr"agt mindestens 1:2.
\begin{kcmt}\begin{komacmt}
	Plenum: "`Braucht wirklich jede Vorlesung eine "Ubung?"' -- Die Antwort ist nat"urlich "`nein"'.
	\emph{Aber}: Wenn "Ubungen angeboten werden (und an anderer Stelle wird ja explizit f"ur Basisvorlesungen
	"Ubungen verlangt). Hier sollte die Formulierung wohl noch "uberarbeitet werden.
\end{komacmt}\end{kcmt}
	\item Zur Kl"arung fachlicher Fragen w"ahrend der Veranstaltung ist ein gewisses Ma"s an Interaktivit"at gegeben. Hierbei werden Thematik und Gruppengr"o"se ber"ucksichtigt. 
	\item Der Vortrag wird fachlich korrekt und sprachlich gut verst"andlich gehalten und ist didaktisch hochwertig.
	\item Eine sich durch das gesamte Semester ziehende Struktur des Lehrstoffes ist klar vom Studenten erkennbar.
	\item Um einen hohen Vernetzungsgrad zwischen den Vorlesungen zu erreichen, gibt es fachliche Einordnungen der Themen und Ausblicke auf weiterf"uhrende Veranstaltungen. 
\end{itemize}

\begin{kcmt}\begin{komacmt}
Nat"urlich sollen alle Veranstaltungen in einer Sprache gehalten werden, der mehrheitlich
die Studenten folgen k"onnen, dies erscheint uns jedoch als selbstverst"andlich.
\end{komacmt}\end{kcmt}



\section{"Ubung/Tutorium}

\begin{kcmt}\begin{komacmt}
	Eine Definition dieser beiden Veranstaltungsformen war notwendig geworden, da sich
	gezeigt hat, da"s unter "`"Ubung"' bzw. "`Tutorium"' an verschiedenen Hochschulen
	Verschiedenes verstanden wird. Momentan ist die Definition noch so gefa"st, da"s
	das Konsens-Verst"andnis (Der Betreuer der "Ubungen ist fachlich h"oher qualifiziert
	als der des Tutoriums, welcher "ublicherweise ein Studierender ist) der Veranstaltungen
	\emph{beide} umfa"st. Das kann sich noch "andern.
\end{komacmt}\end{kcmt}

\subsection{Definition} 

	Eine "Ubung bzw. ein Tutorium ist eine Kleingruppe von allerh"ochstens 30 Studenten, die von einem geeigneten 
	Lehrverantwortlichen betreut wird und notwendigen Stoff und "Ubungsaufgaben behandelt.

\subsection{Ziel} 

	In einer "Ubung bzw. einem Tutorium wird die in der Vorlesung vermittelte Theorie angewandt und wiederholt
	sowie erlernter Stoff gefestigt. "Ubungen und Tutorien besch"aftigen sich mit der Konstruktion von Beispielen 
	und L"osungen von Aufgabenstellungen.

\begin{kcmt}\begin{komacmt}
Der Gedanke hierbei ist, dass "Ubungen sowohl L"osung, L"osungen und auch -- bei "`interessanten"'
Themen mehrere L"osungsm"oglichkeiten aufzeigen und "`vorexerzieren"' sollen.
\end{komacmt}\end{kcmt}

\subsection{Anforderungen}

\begin{itemize}
	\item Die Veranstaltungen sind mit den zugeh"origen Vorlesungen eng verkn"upft.
	\item Der Schwerpunkt liegt auf Interaktivit"at.
	\item Die "Ubungsaufgaben zu den Basisvorlesungen werden korrigiert und kooperativ gel"ost, w"ahrend es bei anderen
		Vorlesungen akzeptabel ist auf vorhandene L"osungen zu verweisen und die autodidaktischen F"ahigkeiten
		der Studierenden zu fordern und f"ordern.
\begin{kcmt}\begin{komacmt}
	In h"oheren Semestern kann man mehr von Studierenden verlangen. Das bedeutet unter anderem
	auch, da"s man von ihnen erwarten kann, da"s sie auch tiefergehende Themen autark aufarbeiten.
	Gleichzeitig soll der Studierende in diesem Proze"s unterst"utzt werden.
\end{komacmt}\end{kcmt}
	\item Im Gro"steil der Zeit sollte die Mehrheit der Studenten in der Lage sein, der "Ubung zu folgen und aktiv mitzuarbeiten.
	\item Zus"atzlich kann eine Global"ubung angeboten werden, die sich auf das Vorrechnen von Aufgaben konzentriert;
		hierbei ist die Gruppengr"o"se nicht beschr"ankt.
\begin{kcmt}\begin{komacmt}
	Dieser Bulletpoint sollte der letzte sein.

	Das Wort \emph{Zus"atzlich} soll hier ausdr"ucken, da"s diese Global"ubungen das oben
	angesprochene Verh"altnis von "Ubungen bzw. Tutorien zu Vorlesungen \emph{nicht} ber"uhren.
\end{komacmt}\end{kcmt}
	
\end{itemize}

\section{Seminar}

\subsection{Definition} 
	In einem Seminar tragen Studenten "uber ein vorher eigenst"andig aufbereitetes Thema vor. Dieses wird
	von einem fachlich kompetenten Lehrk"orper betreut.

\begin{kcmt}\begin{komacmt}
Lehrk"orper = Dozent, weitere Mitarbeiter die Vortr"age betreuen, "`alle, die etwas mit dem Seminar zu tun haben"' (und keine H"orer sind).
\end{komacmt}\end{kcmt}

\subsection{Ziel} 
	Ziel eines Seminars ist es, das eigenst"andige wissenschaftliche Arbeiten zu f"ordern
	und zur Pr"asentation von Ergebnissen zu bef"ahigen. Der Student entwickelt hierbei ein
	tiefergehendes fachliches Verst"andnis.

\begin{kcmt}\begin{komacmt}
"uben. trainieren. aneignen. Formulierungswahn!
\end{komacmt}\end{kcmt}


\subsection{Anforderungen}
\begin{kcmt}\begin{komacmt}
\paragraph{Brainstorm!} \begin{itemize}
	\item Kein Powerpoint! (tongue in cheek)
	\item Vortrag eines Studierenden 
	\item zur Verf"ugung stehende Zeit zur Vorbereitung
	\item Betreuung
	\item Anspruch des Themas
	\item Feedback -- M"oglichkeiten zur (anschlie"senden) Diskussion
	\item Gruppengr"o"se (klein)
	\item Umfang des Stoffes (moderat)
	\item (vorgeschlagen und gestrichen  war noch: ) aufeinander aufbauende Themen
	\item \dots
	\item Eigenst"andigkeit
	\item F"ahigkeit zur Pr"asentation
	\item wissenschaftlich Arbeiten
	\item tieferes VErst"andnis
	\item intensive BEsch"aftigung mit Thema
	\item Interaktion
	\item Zwischenfragen ( + Vorbereitung darauf )
	\item Stil
\end{itemize}
"`Bef"ahigung zur Pr"asentation"': Zusammenfassen, zeitlicher, stofflicher Rahmen.
Auswahl der Tiefe des Themas.
\end{komacmt}\end{kcmt}

\begin{itemize}
	\item Alle Vortr"age beziehen sich auf ein vorher bekanntgegebenes Rahmenthema.
	\begin{kcmt}\begin{komacmt}
		(Fritz) Oberthema durch Rahmenthema ersetzt.
	\end{komacmt}\end{kcmt}
	\item Unterschiedlicher Arbeitsaufwand ist vor Vergabe der Vortr"age bekannt und auf Wunsch anzugleichen.
	\item W"ahrend der Erarbeitungsphase stellt der Dozent einen Ansprechpartner f"ur R"uckfragen zur Verf"ugung.
	\begin{kcmt}\begin{komacmt}
		(Fritz) Nat"urlich kann der Dozent auch sich selbst zur Verf"ugung stellen. Eine Minimalforderung
		ist jedoch "`nur"' einen angemessenen Ansprechpartner vorgesetzt zu bekommen, der vor allem bei
		fachlichen Fragen weiterhelfen kann.
	\end{komacmt}\end{kcmt}
	\item Der Anspruch der Vortragsthemen korreliert mit der zur Verf"ugung stehenden Bearbeitungszeit. Diese
		betr"agt mindestens zwei Wochen.
	\begin{kcmt}\begin{komacmt}
		(Fritz) Also die Zeit die zur Bearbeitung zur Verf"ugung steht betr"agt zwei Wochen, nicht wir
		gehen von einer Arbeitslast von mindestens zwei Wochen aus. Die Formulierung ist noch etwas wacklig m.E.
	\end{komacmt}\end{kcmt}
	\item Die Vortragenden erhalten Feedback vom Dozenten sowie auf Wunsch auch vom Auditorium.
	\item Ein Thema wird maximal von drei Studenten bearbeitet; jeder am Seminar teilnehmende Student hat die M"oglichkeit, an einem Vortrag mitzuwirken und pr"asentiert mindestens eine halbe Stunde lang.
\end{itemize}


\begin{kcmt}\begin{komacmt}
	(Karlsruhe) Julia besteht auf $>1$ Studierende/Vortrag, pr"aferiert 3. 2 tragbar. $<2$ Veto. Fritz will einen Vortragenden pro Thema. Z"ahneknirschender Konsenes mit Verweis auf \emph{Minimal}standards hergestellt.

	(Karlsruhe) Eine Folgerung: Gruppengr"o"se beschr"ankt weil jeder drankommen k"onnen soll, max 2 Leute pro Vortrag $\Rightarrow$ bei X Vorlesungswochen ($X=10\ldots 16$) ergibt sich eine obere Grenze.
	
	(Graz) Erneute Diskussion: Ideale Lerngruppengr"o{"s}e liegt bei 2-3, wir haben eine Mindestvortragszeit von 30 min/Student. Damit ergibt sich bei 90 min Vortrag: max 3 Studenten. Mit Verweis auf \emph{Minimal}standards haben wir auf 3 Studenten abgeschw"acht.
\end{komacmt}\end{kcmt}


	\begin{kcmt}\begin{komacmt}

\section{Vorkurs}

	Wir haben uns entschieden, den Vorkurs zu streichen, weil die Tatsache, dass der Vorkurs nicht z.B. f"ur LinA oder Analysis als Voraussetzung verpflichtend sein darf bereits durch den Abschnitt Freiheit des Lernens abgedeckt ist.
	Falls ein ``Vorkurs'' direkt in der Liste der Voraussetzungen f"ur den Abschluss stehen sollte, so kann man ihn dann eben irgendwann besuchen, so bekloppt das sein mag.

\subsection{Definition}
Ein Vorkurs ist eine Veranstaltung, die darauf ausgelegt ist, vor dem Beginn des eigentlichen Studiums besucht zu werden.

\subsection{Anforderungen}
\begin{itemize}
\item Falls ein Vorkurs stattfindet, darf er nicht verpflichtend sein.
\end{itemize}

	\end{komacmt}\end{kcmt}
