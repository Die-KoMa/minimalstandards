\section{Evaluationsbogen}

\begin{komacmt}
Seite so formulieren / layouten, dass Faxantwort m"oglich.

Dieser Bogen ersetzt nicht den eigentlichen Text!
\end{komacmt}

Bitte kreuzen Sie an, ob die folgenden Aussagen auf den mathematischen Bereich Ihrer Hochschule zutreffen.
Die Aussagen beziehen sich auf den vorliegenden Text bzw. gehen aus ihm hervor.
Bei Unklarheiten ziehen Sie bitte den entsprechenden Abschnitt des Textes zu Rate.

\begin{itemize}
\item Diversit"at und Spezialisierung
 \begin{itemize}
 \item Es werden mindestens zwei Richtungen in der Mathematik angeboten.
 \item gen"ugend Veranstaltungen f"ur Vertiefung in mindestens zwei Richtungen
 \item Engagement und Selbstst"andigkeit von Studenten wird erm"oglicht.
 \end{itemize}

\item Erste Studienphase
 \begin{itemize}
 \item Studenten werden an das mathematische Angebot herangef"uhrt.
 \item Unterschiedliche Vorkenntnisse werden ber"ucksichtigt.
 \item Studenten k"onnen die Hochschule leicht wechseln.
 \end{itemize}

\item Zweite Studienphase
 \begin{itemize}
 \item Gleichwertige Leistungen von anderen Hochschulen werden anerkannt.
 \end{itemize}

\item Dritte Studienphase
 \begin{itemize}
 \item Studenten lernen, selbstst"andig wissenschaftlich zu arbeiten.
 \end{itemize}

\item Kontinuit"at
 \begin{itemize}
 \item Das Studium ist unabh"angig von gew"ahlten Spezialisierungen in der Regelstudienzeit absolvierbar.
 \item Ein Ausfall von einem Semester verl"angert das Studium um h"ochstens zwei Semester.
 \end{itemize}

\item Teilzeitstudium / Studieren mit Kind
 \begin{itemize}
 \item Studieren mit Kind ist m"oglich.
 \item Es gibt daf"ur einen Beispielstudienverlauf, der keinen finanziellen Nachteil nach sich zieht.
 \end{itemize}

\item Pr"ufungen
 \begin{itemize}
 \item Pr"ufungsmodalit"aten werden zu Beginn jeder Veranstaltung transparent dargestellt.
 \item Pr"ufungstermine werden mindestens einen Monat vorher angek"undigt.
 \item Pr"ufungen d"urfen mindestens zweimal wiederholt werden.
 \item Pr"ufungen k"onnen so z"ugig wiederholt werden, dass keine Verz"ogerung im Studienverlauf erfolgt.
 \item Alternativ d"urfen vor einer Pr"ufungswiederholung auch relevante Vorlesungen wiederholt werden.
 \end{itemize}

\item Schriftliche Arbeiten
 \begin{itemize}
 \item Aktive Betreuung von schriftlichen Arbeiten ist vorhanden.
 \item Vergabe des Themas erfolgt zeitnah nach Anfrage.
 \end{itemize}

\item Kontinuit"at der Studienordnung
 \begin{itemize}
 \item Jedem Studenten, der in einer Studienordnung beginnt, ist es m"oglich, in dieser fertigzustudieren.
 \end{itemize}

\item Freiheit des Lernens
 \begin{itemize}
 \item Anwesenheitspflicht existiert h"ochstens bei Seminaren.
 \item Ein autodidaktisches Studium wird nicht ausgeschlossen.
 \item Pr"ufungen besitzen keine Zulassungsvoraussetzungen aus anderen Veranstaltungen.
 \end{itemize}

\item Orientierung an den Interessen der Studierenden
 \begin{itemize}
 \item Studenten sind durch gew"ahlte Vertreter an allen Gremien beteiligt, die Studien- und Pr"ufungsordnungen beschlie"sen.
 \end{itemize}

% Veranstaltungsformen
\item Veranstaltungsformen
 \begin{itemize}
 \item Alle Veranstaltungen der Hochschule sind frei zug"anglich.
 \item Alle Lehrenden sind au"serhalb von Veranstaltungen erreichbar.
 \item Die Lehrenden bzw. Betreuenden sind fachlich und didaktisch kompetent.
 \item Es wird gelehrt, Probleme zu l"osen.
 \item Nachbereitung von Vorlesungen "uberschreitet nicht das eineinhalbfache der Vorlesungszeit.
 \item Alle Veranstaltungen werden jedes Semester von Lehrenden und Studenten evaluiert.
 \item Die Anforderungen aus dem Abschnitt "`Vorlesung"' sind erf"ullt.
 \item Die Anforderungen aus dem Abschnitt "`"Ubung/Tutorium"' sind erf"ullt.
 \item Die Anforderungen aus dem Abschnitt "`Seminar"' sind erf"ullt.
 \end{itemize}

\item Infrastruktur
 \begin{itemize}
 \item Die Anforderungen aus dem Abschnitt "`Veranstaltungsr"aume"' sind erf"ullt.
 \item Die Anforderungen aus dem Abschnitt "`Computerr"aume"' sind erf"ullt.
 \item Die Anforderungen aus dem Abschnitt "`Studentische Arbeitsr"aume"' sind erf"ullt.
 \item Die Fachschaft wird durch die Hochschule unterst"utzt.
 \item Der Fachschaft steht die im Abschnitt "`Fachschaft"' genannte Infrastruktur zur Verf"ugung.
 \item Fachschaftssitzungen werden erm"oglicht.
 \item Alle R"aumlichkeiten sind barrierefrei zug"anglich.
 \item Aufeinander folgende Pflichtveranstaltungen finden r"aumlich nahe genug beieinander statt.
 \item Geb"aude und R"aume sind deutlich sichtbar gekennzeichnet.
 \item Die Anforderungen aus dem Abschnitt "`Bibliothek"' sind erf"ullt.
 \end{itemize}

\item Service
 \begin{itemize}
 \item Die Anforderungen aus dem Abschnitt "`Studienberatung"' sind erf"ullt.
 \item Die Anforderungen aus dem Abschnitt "`Fachberatung"' sind erf"ullt.
 \item Die Anforderungen aus dem Abschnitt "`Betreuung der Studienanf"anger"' sind erf"ullt.
 \item Die Anforderungen aus dem Abschnitt "`Praktikum"' sind erf"ullt (nur relevant f"ur Pflichtpraktika)
 \item Die Anforderungen aus dem Abschnitt "`Webseite"' sind erf"ullt.
 \item Studenten werden zum Auslandsstudium beraten.
 \item Studenten aus dem Ausland werden beim Einfinden in der Hochschule unterst"utzt.
 \end{itemize}

\end{itemize}
