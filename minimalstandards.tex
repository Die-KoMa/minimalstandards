%%%
% Dokumentenshell f"ur den Bericht: AK Minimalstandards in der Lehre.
% FritzTeX! Hier werden/wurden Ergebnisse der verschiedenen Teilnehmer
% des Aks Minimalstandards verwurstet. Der Ak wurde zweimal vom Fritz
% geleitet; das Material stammt lat"urnich von den Teilnehmern.
%
% $Id: shell.tex,v 1.12 2008/05/05 17:18:38 brainbug Exp $
%%%
\documentclass[10pt,twoside,a5paper,openright]{book}
\usepackage[utf8]{inputenc}
\usepackage{ngerman}
\usepackage{url}
\usepackage{fancyhdr}                % Leck're headers
\usepackage{version}
\usepackage{fullpagegraphic}
\usepackage{latexsym}                % symbole!!
\usepackage{amsmath}                 % Mehr mathematische Befehle...
\usepackage{amssymb}                 % ... und Symbole
\usepackage[final,pdftex,colorlinks,urlcolor=blue]{hyperref}		% setzt Links und Verweise im PDF-Dokument
\usepackage{graphicx}               % zum Einf"ugen von eps-Bildern etc.
\usepackage{wasysym}
\usepackage{color,colortbl,titlesec, array}
% Fot Watermark
% \usepackage{draftwatermark}
% \SetWatermarkScale{2}
% \SetWatermarkText{\sf Nach Bremen}

\setlength{\hoffset}{-.5in}
\setlength{\voffset}{-1in}
%\addtolength{\textheight}{1.9in}
%\addtolength{\textwidth}{0.9in}
\setlength{\textheight}{15.5cm}
\setlength{\textwidth}{11.5cm}
\setlength{\oddsidemargin}{0.3cm}
% \setlength{\evensidemargin}{.2cm}
\setlength{\topmargin}{1.3cm}
\setlength{\footskip}{1.3cm}

%%% KOPF & FUSSZEILEN
\pagestyle{fancy}
\fancyhead{} % Aufraeumen
\fancyhead[ER]{\scshape \footnotesize \leftmark}
\fancyhead[OL]{\scshape \footnotesize \rightmark}
\fancyhead[EL,OR]{\thepage}
\fancyfoot{} % Aufraeumen
\fancyfoot[EL]{\scshape \footnotesize KoMa-Handbuch}
%\fancyfoot[EC,OC]{\scshape \footnotesize FH Regensburg}
\fancyfoot[OR]{\scshape \footnotesize Minimalstandards}


\definecolor{grau}{gray}{.6}

\excludeversion{kcmt}
\excludeversion{stilltodo}
\excludeversion{kcmt62}
\excludeversion{kcmt6XX}
%\includeversion{kcmt}
% \includeversion{stilltodo}

\renewcommand{\normalsize}{\small}

\newenvironment{komacmt}{%
\marginpar{\framebox{!}}%
\begin{center}\begin{minipage}{.85\textwidth}\normalfont \mdseries \sffamily \small \color{grau} }{\end{minipage}\end{center}}

\begin{kcmt62}%
\newenvironment{komacmt62}{%
\marginpar{\framebox{62}} %
\begin{center}\begin{minipage}{.8\textwidth}\normalfont \mdseries \sffamily \small \color{grau} }{\end{minipage}\end{center} %
}
\end{kcmt62} %

\newcounter{messgroesse}
\newcommand{\messgroesse}{\addtocounter{messgroesse}{1}$x_{\Roman{messgroesse}}$}


\begin{document}
% Farbiger Titel plus Schmutztitel
% aussen Umschlag
\frontmatter
\includegraphicsfullpage{titelseite}
\newpage
% leere Seite
\thispagestyle{empty}~
\newpage

%%% Schmutzumschlag
\begin{titlepage}
\begin{flushright}\sffamily
	\vspace*{1cm}
	{\Huge{Handbuch}}\\
	\vspace{2.0cm}
	{\large zu}\bigskip\par
	{\huge{Minimalstandards in der Lehre}} \\
	\vspace{2ex}
	\vfill
	{\large{Arbeitskreis Minimalstandards}\smallskip \\ 
	{\large{der Konferenz der deutschsprachigen Mathematikfachschaften}}\\
	\vspace{1cm}
	erarbeitet auf den Konferenzen}\smallskip\\	
	{\small
	Bielefeld (WS 06/07), \\
	Karlsruhe (SS 07),  \\
	Regensburg (WS 07/08), \\
	Chemnitz (SS 08), \\
	Paderborn (WS 08/09),\\
	Graz (WS 09/10)
	}\\
	\vspace{3cm}
	{\Large Wintersemester 2009/10}\\
	\vspace{14ex}
\end{flushright}
\end{titlepage}

%Impressum
\newpage
\vspace*{5cm}

\begin{center}
	Minimalanforderungen an gute Lehre der Mathematik
	an Hochschulen im deutschsprachigen Raum,\\
	vertreten und erarbeitet durch die Konferenz der deutschsprachigen Mathematikfachschaften (KoMa)
\end{center}

\vfill
\subsection*{Impressum}

\begin{table}[h]
\footnotesize
%	\begin{center}
		\begin{tabular}{ll}
		Herausgeber:				& KoMa-B"uro \\
									& c/o Fachschaftsrat Mathematik \\
									& an der TU Chemnitz \\
									& \texttt{www.die-koma.org} \\
		Erschienen:					& Dezember~2009 \\
		Auflage:					& 1.~Auflage\\
		Redaktion:					& XXX \\
									& Arbeitskreis "`Minimalstandards"'\\ 
		Redaktionsschluss:			& \today \\
		Druck:						& XXX \\
												& \\
		Copyright:					& Das Copyright f"ur alle Texte liegt bei den jeweiligen Autoren. \\
		\end{tabular}
%	\end{center}
\end{table}

% Inhaltsverzeichnis
\newpage
\tableofcontents\thispagestyle{fancy}


\setlength{\parskip}{1.5ex}

\begin{kcmt62}
\begin{komacmt62}
Ein paar allgemeine Anmerkungen vom AK-Leiter (Fritz, Fh-Regensburg):
\begin{itemize}
\item der Arbeitskreis trifft sich inzwischen mindestens zum vierten Mal; einige
 Mitglieder sind seit den guten alten Zeiten mit dabei und haben inzwischen ein
 relativ gutes Gesp"ur daf"ur entwickelt, was konsensf"ahig sein k"onnte. Diese
 haben sich auch auf dem Ak gemeldet, und k"onnen als Ansprechpartner f"ur die
 "`Hausaufgaben"' fungieren.
\item Der Inhalt dieses Papiers so wie es bisher existiert, findet eine breite
 Unterst"utzung.
\item Es gibt so ein paar "`Kreisdreher"' -- mal kommt eine Formulierung rein, mal
 wieder raus, alles in allem jedoch immer unter verschiedenen Gesichtspunkten,
 also nein, wir drehen uns nicht nur im Kreis.
\item Wir haben uns entschlossen, dieses Mal alle Zahlen fl"oten gehen zu lassen.
 Einerseits sind zwar konkrete Zahlen w"unschenswert, andererseits beinhalten sie
 auch ein massives Problemfeld, einerseits das der Messbarkeit allgemein, andererseits
 auch der Umgang, der mit diesen Zahlen zu erwarten ist. Beispielsweise die Rechner
 pro Studierenden -- wir forderten mindestens 25 Rechner, ab 250 Studierenden in
 der Fakult"at aber 10:1 Studierende:Rechner. Was nun wenn 260 Studierende in einer
 Fakult"at sind, aber nur 25 Rechner vorhanden ? Ist dies schlecht? Dies kommt darauf
 an, ob die angegeben Zahl die \emph{absolute Schmerzgrenze} ist oder eine Empfehlung
 die eine gewisse Abweichung toleriert. Insgesamt ist es jedoch massiv schwer konkrete
 Zahlen zu erarbeiten, und man macht sich und die gesamte Arbeit des Arbeitskreises
 relativ leicht angreifbar. Dies gilt es zu vermeiden. Stattdessen hat sich dieser
 Arbeitskreis (KoMa62) dazu entschieden alle z"ahlbaren Forderungen variabel zu
 definieren, und in einer Dokumentation der Minimalstandards verschiedene Richtwerte
 zu erw"ahnen, zusammen mit einer Wertung der konkreten Zahlen. So kann man sich
 selbst einsch"atzen wo man auf der Skala liegt.
\end{itemize}
\end{komacmt62}
\end{kcmt62}

\mainmatter

\newpage
% $Id: praeambel.tex,v 1.9 2008/05/26 15:30:13 brainbug Exp $
\section{Pr"aambel}

In den letzten Jahren wurden im deutschsprachigen Raum rege
Debatten "uber die Qualit"at der Lehre gef"uhrt.
Bei diesen war der Einflu"s praxisferner Experten unserer Meinung nach "uberproportional gro"s.
Dadurch fehlten Standpunkte und Erfahrungen derer, die
die Praxis erleben, n"amlich unsere. Als Studierende
aller Semester, aller Hochschularten, aller mathematischen Studieng"ange
und aller auslaufenden oder eingef"uhrten Studiensysteme sehen
wir unsere Standpunkte im Ergebnis der Debatte nur unzureichend
ber"ucksichtigt.

Gleichzeitig muss man feststellen: Die Umsetzung des Reformprozesses
ist hinter den Erwartungen der Hochschulen, Wirtschaft und Politik
zur"uckgeblieben. Die gestiegene Belastung hindert die Studierenden
an der Entfaltung ihrer wissenschaftlichen F"ahigkeiten, worunter
die gesamte Wissenschaftslandschaft leidet und nachhaltig negativ
beinflusst wird.

Getrieben von einem Wunsch nach Eliten werden die Anforderungen der
Masse vernachl"assigt. Insbesondere wird ignoriert oder "ubersehen,
dass es das Ziel der Hochschulen sein muss, nicht nur eine Elite zu
schaffen, sondern das Gros erfolgreich zu Wissenschaftlern auszubilden.
Hierzu m"ussen gewisse Anforderungen von jeder Hochschule erf"ullt
sein, eben die im Folgenden aufgef"uhrten \emph{Minimalstandards}.
Die Bringschuld zur Erf"ullung dieser Minimalstandards liegt sowohl
bei den Hochschulen als auch bei Bund und L"andern, die die
"offentlichen Hochschulen zu finanzieren haben.

\paragraph{Anmerkungen zum Sprachgebrauch}\begin{enumerate}
\item Formulierungen im Pr"asens stellen Forderungen dar, die im
	Rahmen der Minimalstandards zu erf"ullen sind.
\item Zur besseren Lesbarkeit wurde auf die klare Geschlechterunterscheidung
	verzichtet. Die verwendeten Formulierungen richten sich
	jedoch ausdr"ucklich an beide Geschlechter.
\end{enumerate}

\input{veranstaltungsangebot}
\chapter{Veranstaltungsformen}
\thispagestyle{fancy}

\begin{kcmt}\begin{komacmt}
\textbf{M"ogliche Formen}...

\begin{itemize}
	\item Vorlesung: (definiert).
	\item "Ubung: (definiert).
	\item Tutorium: (definiert).
	\item Praktika: (Anforderungen bei Pflichtpraktika unter Infrastruktur zu finden).
	\item Seminar: (definiert)
	\item Fragestunden/Konversatorische Stunden (Braucht's nicht: T + "U + geile Studis = kein Problem).
	\item Nachhilfe (Braucht's nicht: T + "U + geile Studis = kein Problem).
	\item Vorkurs/Orientierung (kein Minimalstandard, n"aheres dazu siehe unten in den Kommentaren)
	\item Service (siehe "`Service"', S.~\pageref{l.service})
\end{itemize}

\end{komacmt}\end{kcmt}

\section{Globale Forderungen}

\begin{itemize}
\item	Alle Veranstaltungen der Hochschule sind frei zug"anglich. 
\begin{kcmt}\begin{komacmt}
	Mit "`frei zug"anglich"' ist gemeint, dass ein Student jede Veranstaltung zumindest als Gasth"orer besuchen darf. Eine aktive Teilnahme kann ggf. bei Veranstaltungsformen wie Seminaren, Praktika auf eine feste Teilnehmerzahl beschr"ankt sein. (neu: Graz KoMa 65)
\end{komacmt}\end{kcmt}

\item	Alle Lehrenden bieten w"ahrend ihrer Lehrveranstaltungen und der Betreuung von Pr"ufungsleistungen eine w"ochentliche Sprechstunde an oder zumindest die M"oglichkeit einen Termin innerhalb einer Woche zu vereinbaren. Sofern der Lehrende f"ur mehr als eine Woche nicht erreichbar ist, muss ein alternativer Ansprechpartner vorhanden sein.

\begin{kcmt}\begin{komacmt}
	(Karlsruhe) Im Teil-AK trat die Frage auf, ob eine garantierte w"ochentliche Sprechstunde
	nicht schon den Rahmen "`Minimal"' sprengt. Reicht nach Vereinbarung
	innerhalb einer Woche nicht schon aus?

	(Karlsruhe: Robert) ist eine w"ochentliche Sprechstunde nicht eine Einschr"ankung?
	Quasi gefordert, es reicht wenn der Prof nur zur Sprechstunde erreichbar ist?

	(Graz) Mit "`Termin innerhalb einer Woche vereinbar"' verstehen wir, dass die Terminabsprache nicht l"anger als eine Woche dauert, der Termin danach aber zeitnah stattfinden muss.
\end{komacmt}\end{kcmt}

\item	Die Lehrenden bzw. Betreuenden sind fachlich und didaktisch kompetent.
	
\item	Es werden nicht nur Probleml"osungen vermittelt. Es wird auch gelehrt, Probleme zu l"osen.

\begin{kcmt}\begin{komacmt}
	(Fritz) Hierbei ist gemeint, da"s nicht nur das reine Reproduzieren von bekannten
	L"osungen ("uberhaupt) eine Lehre der Mathematik auszeichnet, sondern das
	Vermitteln einer mathematischen Denke. Hierzu ist es absolut notwendig,
	da"s neben Probleml"osungen eben auch gelehrt wird Probleme zu l"osen.
\end{komacmt}\end{kcmt}
	
\item	Vom Studenten wird erwartet, den Stoff der vorhergehenden Lehrveranstaltung durch
	Aufbereitung ausreichend verinnerlicht zu haben, um ein kontinuierliches Voranschreiten
	im Stoff zu gew"ahrleisten. Der Zeitaufwand daf"ur "uberschreitet dabei das 
	eineinhalbfache der f"ur die Vorlesung vorgesehenen Zeit nicht.

\item	Die hier vorgestellten Veranstaltungsformen beziehen sich auf alle Phasen des Studiums.
	Der Gebrauch des Begriffes "`Basisveranstaltung"' beschreibt die Veranstaltungen der
	ersten Studienphase.
	
\item	Alle Veranstaltungen werden jedes Semester evaluiert, sofern die Anonymität der Befragten gewährleistet bleibt. Die Ergebnisse sind f"ur die Studenten in ausgewerteter Form zug"anglich.

\begin{kcmt}\begin{komacmt}
	Details zur Evaluation kommen anderswo her und ist auch von der Veranstaltungsform abh"angig.
	Nicht how-to-eval vorschreiben sondern da"s\dots

	(Graz) F"ur Details verweisen wir auf die Resolution der KoMa~64 zu Evaluationen.
\end{komacmt}\end{kcmt}

\item	Nach dem bestandenen ersten Studienabschnitt wird davon ausgegangen, dass alle Studenten 
	sich auf etwa gleichem Niveau befinden. Hierbei wird auch auf Schwankungen bei den
	Vorkenntnissen der Studenten eingegangen, d.h. das erreichte Niveau ist unabh"angig
	vom Zeitpunkt des Studienbeginns. Eventuell vorhandene und erkannte M"angel des Studenten
	werden durch zus"atzliche Veranstaltungen oder Hilfestellungen, wie z.\,B. "Ubungen, Zusatzmaterial ausgeglichen.

\item	F"ur die Mehrheit der Studenten gen"ugen 50 Stunden Arbeitsaufwand pro Woche, um das Studium in Regelstudienzeit erfolgreich abzuschlie"sen.
	Der Arbeitsaufwand beinhaltet sowohl die Zeit f"ur den Besuch von Veranstaltungen als auch f"ur die Nachbereitung, Hausaufgaben, schriftliche Arbeiten und "ahnliches.
\begin{kcmt}\begin{komacmt}
	x bestimmen. (Fritz) Hinweis: Eigentlich gibt es ECTS, nach derem System basierend auf
	Semesterwochenstunden "`ausgerechnet"' werden k"onnte, wieviel Wochenarbeitsstunden auf
	"Ubungsaufgaben entfallen k"onnen. Der Sinn hier ist eine Begrenzung nach oben, und es
	ist fraglich, ob die bearbeitende Gruppe einen Konsens mit dem ECTS findet.
\end{komacmt}\end{kcmt}
\end{itemize}

\section{Vorlesung}

\subsection{Definition} 
	Eine Vorlesung ist eine regelm"a"sige und fortlaufende Unterrichtsveranstaltung, die von einem
	Professor, Lehrbeauftragten oder wissenschaftlichen Mitarbeiter im Vortragsstil gehalten wird.


\subsection{Ziel} 
	Ziel von Vorlesungen ist die Vermittlung fachlichen Wissens auf theoretischer Basis. 

\subsection{Anforderungen} \label{vorlesung:anforderungen}

\begin{itemize}
	\item Der Lehrstoff ist inhaltlich und visuell so aufbereitet, dass die Studenten
	mehrheitlich nicht "uberfordert sind.
\begin{kcmt}\begin{komacmt}
	Hier betreten wir ein Minenfeld, das Spannungsfeld "`Qualit"at der Vorlesungen"' $\Leftrightarrow$ "`Qualit"at
	der Studierenden"'. Man k"onnte die Anforderung, Zwei Drittel der Studierenden nicht zu verlieren, auch
	dadurch erf"ullen, indem der Stoffumfang erheblich gek"urzt wird. Ref. Fachliche Breite und Tiefe :)

	Anders gesagt, der Stoff mu"s gleich bleiben bzw. der Stoffumfang sollte nicht gek"urzt werden
	um hier etwas zu erreichen.

	Weiterhin macht die Messung ein Problem: Gerade zu Beginn des Studiums sind einige Studierende
	noch anwesend, die f"ur das Studium (allgemein oder das der Mathematik) ungeeignet sind. In dieser
	inhomogenen Menge (bzgl. des vorherigen Ausbildungs- und Leistungsstand) eine Messung durchzuf"uhren
	f"uhrt hier am Ziel vorbei.

	Anmerkung aus dem Plenum: diese "`Messung"' kann ja auch von den betreuenden Studierenden
	durchgef"uhrt werden ($\rightarrow$ "Ubungsgruppenleiter, Mentor, \dots).
\end{komacmt}\end{kcmt}
	\item Durch Bereitstellung und/oder Verweise auf begleitende Lehrmaterialien ist es dem Studenten
		m"oglich das Lernziel auch autodidaktisch zu erreichen sowie in der Vorlesung angeeignetes Wissen weiter zu
		vertiefen.
	\item Eine Vorlesung wird bei Basisveranstaltungen grunds"atzlich von "Ubungen und/oder Tutorien begleitet. Das Verh"altnis der Stundenzahl von "Ubungen/Tutorien zur Vorlesung betr"agt mindestens 1:2.
\begin{kcmt}\begin{komacmt}
	Plenum: "`Braucht wirklich jede Vorlesung eine "Ubung?"' -- Die Antwort ist nat"urlich "`nein"'.
	\emph{Aber}: Wenn "Ubungen angeboten werden (und an anderer Stelle wird ja explizit f"ur Basisvorlesungen
	"Ubungen verlangt). Hier sollte die Formulierung wohl noch "uberarbeitet werden.
\end{komacmt}\end{kcmt}
	\item Zur Kl"arung fachlicher Fragen w"ahrend der Veranstaltung ist ein gewisses Ma"s an Interaktivit"at gegeben. Hierbei werden Thematik und Gruppengr"o"se ber"ucksichtigt. 
	\item Der Vortrag wird fachlich korrekt und sprachlich gut verst"andlich gehalten und ist didaktisch hochwertig.
	\item Eine sich durch das gesamte Semester ziehende Struktur des Lehrstoffes ist klar vom Studenten erkennbar.
	\item Um einen hohen Vernetzungsgrad zwischen den Vorlesungen zu erreichen, gibt es fachliche Einordnungen der Themen und Ausblicke auf weiterf"uhrende Veranstaltungen. 
\end{itemize}

\begin{kcmt}\begin{komacmt}
Nat"urlich sollen alle Veranstaltungen in einer Sprache gehalten werden, der mehrheitlich
die Studenten folgen k"onnen, dies erscheint uns jedoch als selbstverst"andlich.
\end{komacmt}\end{kcmt}



\section{"Ubung/Tutorium}

\begin{kcmt}\begin{komacmt}
	Eine Definition dieser beiden Veranstaltungsformen war notwendig geworden, da sich
	gezeigt hat, da"s unter "`"Ubung"' bzw. "`Tutorium"' an verschiedenen Hochschulen
	Verschiedenes verstanden wird. Momentan ist die Definition noch so gefa"st, da"s
	das Konsens-Verst"andnis (Der Betreuer der "Ubungen ist fachlich h"oher qualifiziert
	als der des Tutoriums, welcher "ublicherweise ein Studierender ist) der Veranstaltungen
	\emph{beide} umfa"st. Das kann sich noch "andern.
\end{komacmt}\end{kcmt}

\subsection{Definition} 

	Eine "Ubung bzw. ein Tutorium ist eine Kleingruppe von allerh"ochstens 30 Studenten, die von einem geeigneten 
	Lehrverantwortlichen betreut wird und notwendigen Stoff und "Ubungsaufgaben behandelt.

\subsection{Ziel} 

	In einer "Ubung bzw. einem Tutorium wird die in der Vorlesung vermittelte Theorie angewandt und wiederholt
	sowie erlernter Stoff gefestigt. "Ubungen und Tutorien besch"aftigen sich mit der Konstruktion von Beispielen 
	und L"osungen von Aufgabenstellungen.

\begin{kcmt}\begin{komacmt}
Der Gedanke hierbei ist, dass "Ubungen sowohl L"osung, L"osungen und auch -- bei "`interessanten"'
Themen mehrere L"osungsm"oglichkeiten aufzeigen und "`vorexerzieren"' sollen.
\end{komacmt}\end{kcmt}

\subsection{Anforderungen}

\begin{itemize}
	\item Die Veranstaltungen sind mit den zugeh"origen Vorlesungen eng verkn"upft.
	\item Der Schwerpunkt liegt auf Interaktivit"at.
	\item Die "Ubungsaufgaben zu den Basisvorlesungen werden korrigiert und kooperativ gel"ost, w"ahrend es bei anderen
		Vorlesungen akzeptabel ist auf vorhandene L"osungen zu verweisen und die autodidaktischen F"ahigkeiten
		der Studierenden zu fordern und f"ordern.
\begin{kcmt}\begin{komacmt}
	In h"oheren Semestern kann man mehr von Studierenden verlangen. Das bedeutet unter anderem
	auch, da"s man von ihnen erwarten kann, da"s sie auch tiefergehende Themen autark aufarbeiten.
	Gleichzeitig soll der Studierende in diesem Proze"s unterst"utzt werden.
\end{komacmt}\end{kcmt}
	\item Im Gro"steil der Zeit sollte die Mehrheit der Studenten in der Lage sein, der "Ubung zu folgen und aktiv mitzuarbeiten.
	\item Zus"atzlich kann eine Global"ubung angeboten werden, die sich auf das Vorrechnen von Aufgaben konzentriert;
		hierbei ist die Gruppengr"o"se nicht beschr"ankt.
\begin{kcmt}\begin{komacmt}
	Dieser Bulletpoint sollte der letzte sein.

	Das Wort \emph{Zus"atzlich} soll hier ausdr"ucken, da"s diese Global"ubungen das oben
	angesprochene Verh"altnis von "Ubungen bzw. Tutorien zu Vorlesungen \emph{nicht} ber"uhren.
\end{komacmt}\end{kcmt}
	
\end{itemize}

\section{Seminar}

\subsection{Definition} 
	In einem Seminar tragen Studenten "uber ein vorher eigenst"andig aufbereitetes Thema vor. Dieses wird
	von einem fachlich kompetenten Lehrk"orper betreut.

\begin{kcmt}\begin{komacmt}
Lehrk"orper = Dozent, weitere Mitarbeiter die Vortr"age betreuen, "`alle, die etwas mit dem Seminar zu tun haben"' (und keine H"orer sind).
\end{komacmt}\end{kcmt}

\subsection{Ziel} 
	Ziel eines Seminars ist es, das eigenst"andige wissenschaftliche Arbeiten zu f"ordern
	und zur Pr"asentation von Ergebnissen zu bef"ahigen. Der Student entwickelt hierbei ein
	tiefergehendes fachliches Verst"andnis.

\begin{kcmt}\begin{komacmt}
"uben. trainieren. aneignen. Formulierungswahn!
\end{komacmt}\end{kcmt}


\subsection{Anforderungen}
\begin{kcmt}\begin{komacmt}
\paragraph{Brainstorm!} \begin{itemize}
	\item Kein Powerpoint! (tongue in cheek)
	\item Vortrag eines Studierenden 
	\item zur Verf"ugung stehende Zeit zur Vorbereitung
	\item Betreuung
	\item Anspruch des Themas
	\item Feedback -- M"oglichkeiten zur (anschlie"senden) Diskussion
	\item Gruppengr"o"se (klein)
	\item Umfang des Stoffes (moderat)
	\item (vorgeschlagen und gestrichen  war noch: ) aufeinander aufbauende Themen
	\item \dots
	\item Eigenst"andigkeit
	\item F"ahigkeit zur Pr"asentation
	\item wissenschaftlich Arbeiten
	\item tieferes VErst"andnis
	\item intensive BEsch"aftigung mit Thema
	\item Interaktion
	\item Zwischenfragen ( + Vorbereitung darauf )
	\item Stil
\end{itemize}
"`Bef"ahigung zur Pr"asentation"': Zusammenfassen, zeitlicher, stofflicher Rahmen.
Auswahl der Tiefe des Themas.
\end{komacmt}\end{kcmt}

\begin{itemize}
	\item Alle Vortr"age beziehen sich auf ein vorher bekanntgegebenes Rahmenthema.
	\begin{kcmt}\begin{komacmt}
		(Fritz) Oberthema durch Rahmenthema ersetzt.
	\end{komacmt}\end{kcmt}
	\item Unterschiedlicher Arbeitsaufwand ist vor Vergabe der Vortr"age bekannt und auf Wunsch anzugleichen.
	\item W"ahrend der Erarbeitungsphase stellt der Dozent einen Ansprechpartner f"ur R"uckfragen zur Verf"ugung.
	\begin{kcmt}\begin{komacmt}
		(Fritz) Nat"urlich kann der Dozent auch sich selbst zur Verf"ugung stellen. Eine Minimalforderung
		ist jedoch "`nur"' einen angemessenen Ansprechpartner vorgesetzt zu bekommen, der vor allem bei
		fachlichen Fragen weiterhelfen kann.
	\end{komacmt}\end{kcmt}
	\item Der Anspruch der Vortragsthemen korreliert mit der zur Verf"ugung stehenden Bearbeitungszeit. Diese
		betr"agt mindestens zwei Wochen.
	\begin{kcmt}\begin{komacmt}
		(Fritz) Also die Zeit die zur Bearbeitung zur Verf"ugung steht betr"agt zwei Wochen, nicht wir
		gehen von einer Arbeitslast von mindestens zwei Wochen aus. Die Formulierung ist noch etwas wacklig m.E.
	\end{komacmt}\end{kcmt}
	\item Die Vortragenden erhalten Feedback vom Dozenten sowie auf Wunsch auch vom Auditorium.
	\item Ein Thema wird maximal von drei Studenten bearbeitet; jeder am Seminar teilnehmende Student hat die M"oglichkeit, an einem Vortrag mitzuwirken und pr"asentiert mindestens eine halbe Stunde lang.
\end{itemize}


\begin{kcmt}\begin{komacmt}
	(Karlsruhe) Julia besteht auf $>1$ Studierende/Vortrag, pr"aferiert 3. 2 tragbar. $<2$ Veto. Fritz will einen Vortragenden pro Thema. Z"ahneknirschender Konsenes mit Verweis auf \emph{Minimal}standards hergestellt.

	(Karlsruhe) Eine Folgerung: Gruppengr"o"se beschr"ankt weil jeder drankommen k"onnen soll, max 2 Leute pro Vortrag $\Rightarrow$ bei X Vorlesungswochen ($X=10\ldots 16$) ergibt sich eine obere Grenze.
	
	(Graz) Erneute Diskussion: Ideale Lerngruppengr"o{"s}e liegt bei 2-3, wir haben eine Mindestvortragszeit von 30 min/Student. Damit ergibt sich bei 90 min Vortrag: max 3 Studenten. Mit Verweis auf \emph{Minimal}standards haben wir auf 3 Studenten abgeschw"acht.
\end{komacmt}\end{kcmt}


	\begin{kcmt}\begin{komacmt}

\section{Vorkurs}

	Wir haben uns entschieden, den Vorkurs zu streichen, weil die Tatsache, dass der Vorkurs nicht z.B. f"ur LinA oder Analysis als Voraussetzung verpflichtend sein darf bereits durch den Abschnitt Freiheit des Lernens abgedeckt ist.
	Falls ein ``Vorkurs'' direkt in der Liste der Voraussetzungen f"ur den Abschluss stehen sollte, so kann man ihn dann eben irgendwann besuchen, so bekloppt das sein mag.

\subsection{Definition}
Ein Vorkurs ist eine Veranstaltung, die darauf ausgelegt ist, vor dem Beginn des eigentlichen Studiums besucht zu werden.

\subsection{Anforderungen}
\begin{itemize}
\item Falls ein Vorkurs stattfindet, darf er nicht verpflichtend sein.
\end{itemize}

	\end{komacmt}\end{kcmt}

\chapter{Infrastruktur}
\thispagestyle{fancy}

\section{Generelles}

\begin{itemize}
\item Grundlegende Dinge, wie ausreichende
Beleuchtung, Heizung, Toiletten, Sitzm"oglichkeiten, Platz zum
Schreiben und auch Schreibmaterialien (z.B. Tafeln oder Whiteboards mit zugeh"origem Material) sind vorhanden.

\item Die Infrastruktur ist w"ahrend der Vorlesungszeiten f"ur den Studenten zug"anglich.

\item Alle R"aumlichkeiten sind barrierefrei zug"anglich.

\item Aufeinander folgende Pflichtveranstaltungen finden 
		nahe genug beieinander statt. Es ist also in der Zeit zwischen den 
		Veranstaltungen m"oglich,  von einem Veranstaltungsort zu dem der folgenden zu gelangen.

\item Geb"aude und R"aume sind deutlich 
		sichtbar (auch international verst"andlich) gekennzeichnet.
		An zentralen Stellen sind Pl"ane vorhanden.
\end{itemize}


\section{R"aume}

\subsection{Veranstaltungsr"aume}

\begin{kcmt}\begin{komacmt}
Vorlesungs- und Seminarr"aume unterscheiden sich nur in der Gr"o"se und 
werden deshalb nicht gesondert behandelt. Spezielle R"aume f"ur Tutorien u.~"a. 
werden hier nicht erw"ahnt, da diese nicht unbedingt erforderlich sind 
(Jede "Ubung kann auch in einem Vorlesungs-/Seminarraum stattfinden.). 
Wenn es extra "Ubungsr"aume g"abe, w"are die Anzahl der insgesamt 
ben"otigten R"aume gr"o"ser (kein Minimalstandard).
\end{komacmt}\end{kcmt}
\begin{itemize}
	\item F"ur jede Veranstaltung steht ein Raum zur Verf"ugung.
	\item Jeder Zuh"orer bekommt bei den Veranstaltungen einen daf"ur vorgesehenen Sitzplatz.
	\item Auch zu Sto"szeiten sind ausreichend Kapazit"aten an R"aumlichkeiten vorhanden.
	\item Die R"aume verf"ugen "uber eine Tafel, die so gro"s ist,
		dass bei einer f"ur alle Anwesenden lesbar gro"sen Anschrift gen"ugend Tafelfl"ache vorhanden ist, um den f"ur das Verst"andnis des aktuellen Themas notwendigen Kontext zu fassen.
	\item Die R"aumlichkeiten m"ussen die M"oglichkeit der Visualisierung per Beamer 
		und/oder Overheadprojektor bieten. Es gibt in jedem Raum eine Projektionsfl"ache. 
		Dazu ist jeder Raum (mindestens die H"alfte aller R"aume gleichzeitig) mit den ben"otigten Ger"aten versorgbar.
\begin{kcmt}\begin{komacmt}
	Es sind nicht zu wenig, weil fast nie alle R"aume gleichzeitig besetzt sind und auch
	f"ur viele Veranstaltungen kein Beamer / Overheadprojektor n"otig ist. Es sind nicht
	zu viel, da es nicht sein kann, da"s sich ein Vortragender in der Wahl der Visualisierung
	nach dem Vorhandensein von Beamer/Overheadprojektor richten mu"s.
\end{komacmt}\end{kcmt}
	\item In den R"aumen ist der Dozent "uberall zu verstehen,
		geeignete Hilfsmittel (z.B. ein Mikrofon) stehen bei Bedarf zur Verf"ugung.
	\item Es gibt Platz, um Jacken, Taschen usw. abzulegen.
\end{itemize}

\subsection{Computerr"aume}
\begin{itemize}
	\item Es gibt eine der Anzahl der Studenten angemessene Menge frei zug"anglicher Rechnerpl"atze.
\begin{kcmt}\begin{komacmt}
	Eine genauere Kl"arnung von "`zug"anglich"' wurde gew"unscht, aber auf eine Definition konnten wir uns noch nicht einigen. Es geht dabei z.B. um "Offnungszeiten, Wartezeiten, et al. Ein Vorschlag war, eine angemessene Anzahl an Rechnerpl"atzen dar"uber zu definieren, dass es Zeiten gibt, zu denen nicht gewartet werden muss.

	\paragraph{Gestrichene Formulierung:} \begin{itemize}
	\item Das Verh"altnis von Studierenden zu Rechnerpl"atzen gesamt unterschreitet 7:1 nicht.
	\end{itemize}

	"`Warum 7:1?"' -- Diese Zahl wurde aus der Situation an der Fachhochschule Regensburg,
	TU M"unchen und Chemnitz gefolgert. Die urspr"ungliche Forderung war hier mal 5:1.
	Das Plenum findet aber auch 10:1 akzeptabel. Hier stellt sich auch die Frage nach
	Me"sgr"o"sen allgemein: sollen sie minimal, maximal, exakt beschreiben? "`Wenn diese
	Zahl "uberschritten wird ist's \emph{definitiv} schlecht?"'

	Die neue Formulierung dr"uckt aus, da"s kleine Fakult"aten mit einer Anzahl von
	Rechner im Verh"altnis 10:1 deutlich zu wenig Rechner h"atten.
\end{komacmt}\end{kcmt}
	\item W"ahrend Veranstaltungen mit PC-Einsatz gibt es mindestens halb so viele Rechner wie Studenten.
	\item Die Rechner sind entsprechend der Richtung der Hochschule mit Software 
		ausgestattet (Computeralgebrasystem, Numerische Software, Statistikprogramm, \dots). 
		Ein Programm zum Anfertigen auch umfangreicherer mathematischer Texte ist installiert.
\begin{kcmt}\begin{komacmt}
	Diese Ausstattung ist nat"urlich einwandfrei lizensiert, legal erworben etc.
	Ob hier auch Lizenzen f"ur Spezialsoftware an Studenten herausgegeben werden
	wurde nicht explizit diskutiert (Rauschen aus dem Plena).
	Freeware bzw. Open Source Software sind nat"urlich auch erlaubt.
\end{komacmt}\end{kcmt}
	\item Die Rechner verf"ugen "uber einen Internetzugang. 
	\item Es gibt eine Druckm"oglichkeit h"ochstens zum Selbstkostenpreis. 
		Die Funktionsf"ahigkeit dieser ist gew"ahrleistet (Toner, Papier vorhanden).
	\item Es gibt eine M"oglichkeit zur Visualisierung bei Lehrveranstaltungen (z.~B. Beamer, Tafel...).
\end{itemize}

\subsection{Studentische Arbeitsr"aume}
\begin{itemize}
	\item F"ur den Studenten besteht die M"oglichkeit freie Kapazit"aten herauszufinden 
		(Raumbelegungsplan) und diese zu nutzen. 
	\item Es ist ein Ruhebereich vorhanden, in dem gearbeitet werden kann.
	\item Ein Student, der an seiner Abschlussarbeit arbeitet, hat immer Zugang zu einem Rechner. 
		(Es sollte immer mindestens ein Rechnerpool frei sein.)
	\item Studentische Arbeitsr"aume sind mit einer Tafel bzw. einem Whiteboard ausgestattet.
\end{itemize}


\section{Fachschaft}

Die Fachschaft ist ein Zusammenschluss von Studenten eines Faches zwecks gemeinsamer Interessenvertretung.
Die Fachschaft wird durch die Hochschule unterst"utzt.

\begin{kcmt}\begin{komacmt}
Es geht hier \emph{nicht} um die Aufgaben der Fachschaften.

\begin{itemize}
	\item Rechnerzugang inklusive Webspace \& Mail-Adresse(n)
	\item Kopierm"oglichkeit
	\item B"uroraum mit Telefon
	\item M"oglichkeit f"ur regelm"a"sige FS-Sitzungen
	\item Erm"oglichung der Herausgabe von Infomaterial
\end{itemize}
\end{komacmt}\end{kcmt}

Um eine effiziente Fachschaftsarbeit zu gew"ahrleisten, stellt die Hochschule der Fachschaft folgendes zur Verf"ugung:
\begin{itemize}
	\item einen B"uroraum mit Telefon, dessen Gr"o"se der Anzahl der Studenten angemessen ist,
	\item einen Rechnerzugang inklusive Webspace f"ur eine Fachschaftswebseite und
		eine E-Mail-Adresse sowie
	\item eine Kopierm"oglichkeit.
\end{itemize}

Au"serdem erm"oglicht die Hochschule der Fachschaft regelm"a"sige
Fachschaftssitzungen (durch Bereitstellen eines geeigneten Raumes) und die
Herausgabe von Infomaterial, z.\,B. den Druck eines regelm"a"sigen Infohefts, von Plakaten,
Flyern und "ahnlichem. 

\newpage
\section{Bibliothek}
Eine Bibliothek ist vorhanden und bietet mindestens folgende M"oglichkeiten:
\begin{itemize}
	\item ausreichende Recherchem"oglichkeiten (z.B. am Rechner)
	\item den Veranstaltungen zu Grunde liegende und vertiefende Literatur
	\item ein Kopierer
	\item Arbeitspl"atze
	\item Die wichtigsten Fachzeitschriften sind vor Ort vorhanden, die anderen 
		per Fernleihe beziehbar.
	\item Nicht vorhandene B"ucher sind per Fernleihe zu beziehen.
	\item Es findet regelm"a"sig eine "Uberpr"ufung des Bedarfs statt, 
		so dass bei h"aufig vergriffenen Werken der Bestand aufgestockt wird.
\end{itemize}

\chapter{Service}
\thispagestyle{fancy}
\label{l.service}

\begin{kcmt62}
\begin{komacmt62}
	Service bezieht sich nicht speziell auf einzelne Veranstaltungen, sondern
	aufs Studium allgemein. Wenn wegen Bachelor und Master die Mobilit"at
	zwischen Hochschulen und/oder L"andern steigt, ist die Beratung ein
	Minimalstandard!

\end{komacmt62}
\end{kcmt62}
\begin{kcmt}\begin{komacmt}
	\paragraph{Bearbeitete Punkte:}
	\begin{itemize}
		\item Studienberatung / Fachberatung
		\item Erstsemesterinformation
		\item Fachschafts-Service
		\item Auslandsangebot
	\end{itemize}

	\paragraph{Ausstehende Punkte:}
	\begin{itemize}
		\item Pr"a-Studierenden-Info
		\item Schulungen f"ur Studentische Hilfskr"afte (Tutoren etc.)
		\item "Offnungszeiten (Bibliothek, Sekretariat)
	\end{itemize}

	Service bezieht sich nicht speziell auf einzelne Veranstaltungen, sondern
	aufs Studium allgemein. Wenn wegen Bachelor und Master die Mobilit"at
	zwischen Hochschulen und/oder L"andern steigt, ist die Beratung ein
	Minimalstandard!
\end{komacmt}\end{kcmt}

\section{Studienberatung}

\begin{kcmt}\begin{komacmt}
\emph{Hochschulweite Studienberatung}
\begin{itemize}
	\item bei Fachfragen sofort \& richtig (Fachberatung) weiterleiten
	\item generellen "Uberblick bieten
\end{itemize}
\end{komacmt}\end{kcmt}

Es gibt eine hochschulweite Beratungszentrale, die kompetent ber"at und weiterleitet.
Das Beratungsangebot umfasst folgende Bereiche:
\begin{itemize}
	\item fachliche Beratung "uber die einzelnen Studieng"ange
	\item Studienfinanzierung
	\item Studienrechtsberatung
	\item Beratung f"ur
		\begin{itemize}
			\item behindertengerechtes Studium
			\item Studenten mit Kind
			\item ausl"andische Studenten
			\item Auslandsstudium
		\end{itemize}
\end{itemize}

Unter angemessener finanzieller und organisatorischer Unterst"utzung kann ein 
Teil der Beratungsverpflichtung an die organisierte Studierendenschaft abgetreten werden.

\begin{kcmt}\begin{komacmt}
	Das hei"st aber auch, dass falls Bereiche der Beratung an die
	Studierendenschaft abgetreten werden, muss die abtretende Stelle
	auch daf"ur Sorge tragen, dass die Beratung auch effizient
	durchgef"uhrt wird.
\end{komacmt}\end{kcmt}

\section{Fachberatung}

\begin{kcmt}\begin{komacmt}
\textbf{Fachberatung}
\begin{itemize}
	\item Studienplanung
	\item Pr"ufungsplanung
	\item "Uberblick "uber m"ogliche Studienvertiefung(en)
	\item Anerkennung von Leistungen von anderen Hochschulen
	\item Informationen zum Studienwechsel
\end{itemize}
\end{komacmt}\end{kcmt}

Die Fachberatung ist daf"ur zust"andig, dass ein Student sein Studium
zielgerichtet durchf"uhren kann. Sie muss insbesondere zu folgenden
Themen kompetent beraten k"onnen:
\begin{itemize}
	\item Studienplanung
	\item Pr"ufungsplanung
	\item Studienvertiefung(en)/Spezialisierung(en)
	\item g"angige Nebenf"acher
	\item Anerkennung von Leistungen, die an anderen Hochschulen erbracht wurden
	\item Studienwechsel, sowohl Wechsel des Studiengangs als auch der Hochschule
\end{itemize}

Innerhalb der Vorlesungszeit ist eine Beratung sp"atestens eine Woche nach
Anfrage eines Studenten gew"ahrleistet. In der vorlesungsfreien Zeit
kann diese Frist auf allerh"ochstens drei Wochen verl"angert werden.

\section{Betreuung der Studienanf"anger}

Jeder Studienanf"anger wird vor Studienbeginn "uber das Informationsveranstaltungsangebot orientiert.
Dieses beinhaltet ein Infoheft sowie die M"oglichkeit einer pers"onlichen Beratung.

\begin{description}
	\item [Beratung] Die persönliche Beratung bietet vor allem einen Ausblick "uber das Studium, informiert "uber Voraussetzungen und Fristen und leitet bei weiterf"uhrenden Fragen an die entsprechenden Beratungsstellen weiter.

	\item [Infoheft] Das Infoheft (digital oder gedruckt) beinhaltet mindestens folgende Punkte:
	\begin{itemize}
		\item Pflichtveranstaltungen des ersten Jahres 
		\item Vorlesungskommentar zu Vorlesungen des ersten Semesters
		\item Wichtige Ansprechpartner bzw. Anlaufstellen (mit Telefonnummer,
			E-Mail, Raumnummer, Sprechzeiten wenn m"oglich)
		\item wichtige Termine (z.\,B. R"uckmeldungsfristen, Pr"ufungsanmeldungszeitr"aume)
		\item Infrastruktur (Lageplan, Rechnerzugang, "Offnungszeiten, Bibliothek)
	\end{itemize}

	\begin{kcmt}\begin{komacmt}
	Voraussetzungen \& Fristen: Was muss ich vor dem Studium noch leisten,
	wof"ur mich noch anmelden.
	\end{komacmt}\end{kcmt}
\end{description}

\section{Praktikum}

Falls in einem Studiengang ein Pflichtpraktikum vorgesehen ist, bietet die Hochschule eine Anlaufstelle f"ur Praktikumsbelange. Diese erm"oglicht eine zeitnahe Betreuung des Studenten. Insbesondere bietet sie regelm"a"sige Sprechzeiten an.

Die Anlaufstelle dient der Vermittlung von Praktikumsstellen. Sie bietet Informationen und Hilfestellung bei eventuellen Problemen. Dies umfasst Informationen rein organisatorischer Art wie auch praktikumsvorbereitende und praktikumsnachbereitende Informationen und Hilfestellungen.

Des Weiteren besitzt die Anlaufstelle Informationen zu eventuellen Fristen, Terminen oder Zulassungsvoraussetzungen und setzt Betroffene fr"uhzeitig dar"uber in Kenntnis.

\begin{kcmt}\begin{komacmt}
gel"oscht, weil in anderen Abschnitten bereits vorhanden

\section{Transparenz}
	\begin{itemize}
		\item "Uberblick "uber Vertiefungen in der Mathematik geben
		\item Was an meiner Hochschule, was wo anders
		\item Kooperation mit anderen Hochschulen
		\item Minimum "Uberblick auf Homepage mit Links
		\item besser Vortr"age, Ringvorlesung o.\,"a.
	\end{itemize}


Die Hochschule stellt jedem Studenten einen inhaltlichen "Uberblick "uber die m"oglichen
Vertiefungen in der Mathematik zur Verf"ugung. Hierbei sind die an dieser
Hochschule angebotenen Vertiefungen ausf"uhrlich dargestellt.

	Eine blo"se Aufz"ahlung reicht hier nicht.
	Wie kann das transportiert werden?
	\begin{itemize}
		\item Homepage
		\item Infoheft
		\item Vortr"age etc.
	\end{itemize}
\end{komacmt}\end{kcmt}

\section{Webseite}

Die Hochschule besitzt eine "offentlich zug"angliche, barrierefreie Webseite, "uber die mindestens folgende Informationen verf"ugbar sind:
\begin{itemize}
\item Aktuelle Pr"ufungsordnungen aller an der Hochschule existierenden Studieng"ange. Dazu geh"oren auch auslaufende Studieng"ange.

\item angebotene Lehrveranstaltungen inklusive Name des Dozenten, kurzer Beschreibung und etwaiger inhaltlicher Voraussetzungen

\item Liste der Lehrst"uhle und Professoren mit angegebener Kontaktm"oglichkeit

\item Verweis auf die Fachschaft und deren Webseite, sofern vorhanden

\item Verzeichnis der vorhandenen Servicestellen

\item wichtige Termine, wie zum Beispiel R"uckmeldefristen, Vorlesungszeiten, Pr"ufungsanmeldungszeitr"aume und -termine, Termine von Informationsveranstaltungen
\end{itemize}


\section{Auslandsangebot}

Die Hochschule bietet dem Studenten die M"oglichkeit eines Auslandsstudiums. Hierbei
unterst"utzt sie den Studenten bei der Wahl und dem Kontakt zu einer Austauschhochschule.

Ein Student, der von einer ausl"andischen Hochschule kommt, wird bez"uglich
Visa und anderen Rechtsfragen, Wohnungssuche, Integration und f"ur ihn geeignete Veranstaltungen
beraten und unterst"utzt.


\newpage
\thispagestyle{empty}
~
\vfill
\begin{center}
\url{www.die-koma.org}
\end{center}

\end{document}
