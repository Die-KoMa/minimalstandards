%%%
% $Id: shell.tex,v 1.7 2007/05/23 15:32:33 phaeton Exp $
%%%
\documentclass[oneside,a4paper,pdftex]{article}
%% Pakete
\usepackage{ngerman}
\usepackage{fancyhdr}
\usepackage{graphicx}
\usepackage[usenames]{color}
\usepackage[colorlinks,linkcolor=blue]{hyperref}
%% Standard:
\tolerance=20000
\setlength{\emergencystretch}{20pt}
\setcounter{secnumdepth}{-1}
%% Mehr Platz:
\setlength{\hoffset}{-.5in}                                                     
\setlength{\voffset}{-1in}                                                      
\addtolength{\textheight}{2in}                                                  
\addtolength{\textwidth}{1in}                                                   
%% Definitionen:

\definecolor{grau}{gray}{.6}

\newenvironment{komacmt}{%
\marginpar{\framebox{!}}%
\begin{center}\begin{minipage}{.85\textwidth}\normalfont \mdseries \sffamily \small \color{grau} }{\end{minipage}\end{center}}

%% Titel
\title{Minimalstandards in der Lehre -- Protokoll des AKs auf der KoMa 61 in Regensburg (WS 07/08)}
\author{Martin ``Fritz'' Weber}
\date{\today}
%% Und los geht's.
\begin{document}

\maketitle

\section{Der AK}

Der Arbeitskreis "`Minimalstandards in der Lehre"' entstand nach Anregung auf der KoMa
in Oldenburg in Bielefeld und wurde in Karlsruhe fortgesetzt. Entstanden ist der Wunsch
nach dem Arbeitskreis in einem Umfeld "`ringsum"' eingef"uhrter Studienbeitr"age (bzw.
-geb"uhren) wegen der Bedingung, da"s sich durch die Erhebung der Beitr"age (Geb"uhren)
die Situation in der Lehre verbessern soll. Gleichzeitig kann es aber nicht sein, da"s
erst durch die Beitr"age (Geb"uhren) eine akzeptable Lehre (in der Mathematik) resultiert.
Daher ist es Ziel des Arbeitskreises, eine Liste von Forderungen zu erstellen. 
Diese Liste soll das Minimalziel "`gerade noch gut zu nennende Lehre"' (in der Mathematik) 
darstellen.



\end{document}
